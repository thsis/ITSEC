We collect the basic descriptive metrics that allow getting the initial insights about the network structure and interactions between nodes.
Part of these metrics describe distributions as well as average, median, minimum and maximum values for:
\begin{itemize}
	\item value of a transaction,
	\item amount of gas used in a transaction,
	\item fee paid for a transaction,
	\item number of transactions per block,
	\item value of transactions per block, etc.
\end{itemize}

Another part follows the changes that happen to the network across time. Those are:
\begin{itemize}
	\item number of transactions over time,
	\item volume of transactions over time,
	\item average fees paid per transaction over time, etc.
\end{itemize}

We associate transactions with individual addresses.
This helps to identify:
\begin{itemize}
	\item major senders and receivers of transactions (both in terms of the number of transactions and monetary value exchanges in these transactions),
	\item net balances of accounts in a studied period of time,
	\item distribution of the net balances across the network,
	\item percentage of wealth accumulated and transacted by major nodes, etc.
\end{itemize}

In order to understand our findings in more detail we try to find out identities of the major nodes using data of the Ethereum scanners such as Etherscan.io, Etherchain.org, Blockchair.com,Bloxy.info, etc.

When relevant (for example in case of transactions-outliers, atypical behavior of nodes, discrepancies in data, etc.) we also consider Ethereum forums, posts and newsfeed in order to get additional information about events and actors that may explain or discard certain findings.